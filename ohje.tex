%% Muunnos markdown'iksi:
%% pandoc -f latex -t markdown_strict ohje.tex


\documentclass[12pt]{article}
\usepackage{fontspec}
\usepackage{polyglossia}
\setdefaultlanguage{finnish}
\usepackage{csquotes}
\usepackage[lastpage,user]{zref}
\usepackage{fancyhdr}
\usepackage{splitidx}\makeindex
\usepackage{url}

% Nämä ovat parametrien oletusarvoja eli saman voi sanoa näin: \usepackage{voikkoindex}
\usepackage[langcode="fi",
            path="/usr/local/lib/voikko",
            nimet=Henkilöhakemisto,
            paikat=Paikannimihakemisto,
            sanat=Asiahakemisto]{voikkoindex}

\SetExtraSurname{Gardie}
\SetExtraSurname{Wirta|nen}
\SetSurnameIndex{Wirtanen}{Virtanen}

\frenchspacing
\headheight 15pt
\raggedright

\lhead{Voikkoindexin asennus- ja käyttöohje}
\rhead{\thepage(\zpageref{LastPage})}
\cfoot{}
\begin{document}
\pagestyle{fancy}
\setlength{\parindent}{0pt}
\setlength{\parskip}{1ex plus 0.5ex minus 0.2ex}

Voikkoindex on LuaLaTeX-paketti, joka tekee hakemistoja ja osaa
taivuttaa suomen kielen sanoja.

Voikkoindex on vapaa ohjelma, jota voidaan levittää ja muuttaa GNU
General Public License'n versiolla 3 tai uudemmalla.


Voikkoindex is a LuaLaTeX package for texts written in Finnish.

Voikkoindex is free software: you can redistribute it and/or modify it
under the terms of the GNU General Public License as published by the
Free Software Foundation, either version 3 of the License, or (at your
option) any later version.



\section*{Käyttö LaTeX-tiedostoissa}

\begin{verbatim}
\usepackage[langcode="fi",
            path="/usr/local/lib/voikko",
            nimet=Henkilöhakemisto,
            paikat=Paikannimihakemisto,
            sanat=Asiahakemisto]{voikkoindex}
\end{verbatim}

\verb=langcode= pitää olla joko \verb="fi"= tai \verb="fi-x-sukija"=
Huomaa, että parametrissa pitää olla lainausmerkit alussa ja lopussa.

\verb=path= on hakemisto, jossa Voikon sanastotiedostot ovat.
Lainausmerkkien pitää olla myös tämän parametrin alussa ja lopussa.

\verb=nimet=, \verb=paikat= ja \verb=sanat= ovat hakemistojen nimet,
jotka splitidx'n komento \verb=\printindex= tulostaa.

Nuo ovat oletusarvoja eli jos ne kelpaavat, voidaan kirjoittaa näin

\begin{verbatim}
\usepackage{voikkoindex}
\end{verbatim}

Siis \verb=voikkoindex= määrittelee kolme indeksiä.


\bigskip
LaTeXissa on paketti \verb=splitidx=, jolla voi tehdä useita
indeksejä, esimeriksi

\begin{verbatim}
Maija Virtaselle\sindex[nimet]{Virtanen, Maija} tuli
Helsingistä\sindex[paikat]{Helsinki}
kirjoja\sindex[sanat]{kirja}
Johan Wolfgang von
Goethesta.\sindex[nimet]{Goethe, Johan Wolfgang@von Goethe, Johan Wolfgang}
\end{verbatim}

Typerä lause, mutta se on vain esimerkki.

LuaLaTeX-paketti voikkoindex muuttaa sanat perusmuotoon ennen niitten
laittamista hakemistoon, joten lause voidaan kirjoittaa

\begin{verbatim}
\VXN{Maija}{Virtaselle} tuli
\VXP{Helsingistä}
\VXS{kirjoja}
\VXVON{Johan Wolfgang}{von}{Goethesta}[.]
\end{verbatim}


\section*{Komennot}

Voikkoindex määrittelee 17 komentoa

\begin{verbatim}
\newcommand{\vxn}[1]{\sindex[nimet]{#1}}
\newcommand{\vxp}[1]{\sindex[paikat]{#1}}
\newcommand{\vxs}[1]{\sindex[sanat]{#1}}
\newcommand{\vxvon}[3]{\vxn{#3, #1@#2 #3, #1}}

\NewDocumentCommand{\VXN}{m m o}
\NewDocumentCommand{\VXP}{m o}
\NewDocumentCommand{\VXS}{m o}
\NewDocumentCommand{\VXVON}{m m m o}

\NewDocumentCommand{\VXA}{m o}
\NewDocumentCommand{\VXI}{m o}
\NewDocumentCommand{\VXY}{m o}
\NewDocumentCommand{\VXL}{m o}
\NewDocumentCommand{\VXF}{m m o}

\NewDocumentCommand{\SetExtraSurname}{m}
\NewDocumentCommand{\SetExtraPlaceName}{m}
\NewDocumentCommand{\SetExtraWord}{m}

\NewDocumentCommand{\SetSurnameIndex}{m m}
\NewDocumentCommand{\SetPlaceNameIndex}{m m}
\NewDocumentCommand{\SetWordIndex}{m m}
\end{verbatim}


\section*{Pienet komennot}

Pienellä kirjoitetut komennot eivät muuta sanoja perusmuotoon, ja ne
on määritelty vain kirjoittamistyön vähentämiseksi. Esimerkkilause
voidaan kirjoittaa myös näin

\begin{verbatim}
Maija Virtaselle\vxn{Virtanen, Maija} tuli
Helsingistä\vxp{Helsinki}
kirjoja\vxs{kirja}
Johan Wolfgang von
Goethesta.\vxvon{Johan Wolfgang}{von}{Goethe}[.]
\end{verbatim}


\section*{von ja muut sellaiset}

Joissakin sukunimissä on osia, joita ei oteta huomioon
aakkosjärkestyksessä. Niitä varten on komennot

\begin{verbatim}
\newcommand{\vxvon}[3]{\vxn{#3, #1@#2 #3, #1}}
\NewDocumentCommand{\VXVON}{m m m o}
\end{verbatim}

Ensimmäisessä parametrissa ovat etunimet, toisessa nimen osat joita
ei oteta huomioon aakkosjärjestyksessä, ja kolmannessa sukunimi.
Esimerkiksi

\begin{verbatim}
\VXVON{Magnus Gabriel}{de la}{Gardieta} muistellaan.
\VXVON{Nils}{af}{Ursinin} nimi voidaan kirjoittaa näin.
\end{verbatim}

Vain sukunimi muutetaan perusmuotoon ennen hakemistoon laittamista.


\section*{Välimerkit}


Jos indeksoitava sana loppuu pisteeseen tai muuhun välimerkkiin,
hakemistokomento on laitettava välimerkin jälkeen tai sanan ja
välimerkin väliin tulee ylimääräistä tyhjää. Sen näkee helpoimmin
y-kirjaimesta

\begin{verbatim}
\VXS{syksy}.
\VXS{syksy}[.]
\end{verbatim}

%\noindent
%\VXS{syksy}. \\
%\VXS{syksy}[.]

Tätä varten voikkondexin komennoissa on ehdollinen parametri, johon
välimerkin voi laittaa.


\section*{Sukunimet ja paikannimet}

\begin{verbatim}
\VXN{<etunimet>}{<sukunimi>}[<välimerkki>]
\VXP{<paikannimi>}[<välimerkki>]
\end{verbatim}

Etunimiä ei muuteta perusmuotoon, mutta sukunimi muutetaan.

Paikannimessä voi olla useampi kuin yksi sana. Tällöin vain viimeinen
sana muutetaan perusmuotoon ja muut pysyvät ennallaan, esimerkiksi

\begin{verbatim}
\VXP{Etelä-Suomen läänissä} =>
  Etelä-Suomen läänissä\vxp{Etelä-Suomen lääni}
\end{verbatim}

\section*{Asiasanat}

\begin{verbatim}
\VXA{<asiasana(t)>}[<välimerkki>]
\VXI{<asiasana(t)>}[<välimerkki>]
\VXS{<asiasana(t)>}[<välimerkki>]
\VXY{<asiasana(t)>}[<välimerkki>]
\end{verbatim}

\verb=\VXA= toimii kuten \verb=\VXS=, mutta muuttaa kaikki kirjaimet
pieniksi kirjaimiksi. A niin kuin alas.

\verb=\VXI= toimii kuten \verb=\VXS=, mutta muuttaa sanan
ensimmäisen kirjaimen isoksi kirjaimeksi ja muut pieniksi.
I niin kuin isolla.

\verb=\VXS= muuttaa sanan perusmuotoon, mutta ei muuta kirjainten kokoa.

\verb=\VXY= toimii kuten \verb=\VXS=, mutta muuttaa kaikki kirjaimet
isoiksi kirjaimiksi. Y niin kuin ylös

Jos asiasanassa on useampi kuin yksi sana, vain viimeinen muutetaan
perusmuotoon, esimerkiksi

\begin{verbatim}
\VXS{auringon maassa}[,] =>
  auringon maassa[,]\vxs{auringon maa}
\end{verbatim}

Mikäli tämä ei ole tarkoitus, käytä komentoa \verb=\VXF=.


\section*{Listat}

\begin{verbatim}
\VXF{<sanat>}{<formaatti>}[<välimerkki>]
\end{verbatim}

Esimerkiksi

\begin{verbatim}
\VXF{Helsingin uudelle nuorisoseuralle}{p=,s,s} =>
  Helsingin\vxp{Helsinki} uusi
  nuorisoseura\vxs{Helsingin uusi nuorisoseura}

\VXF{Helsingin seudun nuorisoseurassa}{p=,=,s}
  Helsingin\vxp{Helsinki} seudun
  nuorisoseurassa\vxs{Helsingin seudun nuorisoseura}
\end{verbatim}

Ensimmäinen parametri on lista sanoja ja toinen parametri sanojen
formaatti pilkuilla toisistaan erotettuina. Sallitut formaatit ovat

\begin{verbatim}
p=  =
ps  s
pa  a
pi  i
py  y
\end{verbatim}

Jos formaatin ensimmäinen kirjain on \verb=p= sana oletetaan
paikannimeksi ja se laitetaan paikannimihakemistoon perusmuodossa.
Muut formaatit ovat

\begin{verbatim}
= sanalle ei tehdä mitään
s sana muutetaan perusmuotoon
a perusmuoto muutetaan pieniksi kirjaimiksi (alas)
i perusmuodon eka kirjain muutetaan isoksi, muut pieniksi (isolla)
y perusmuoto muutetaan isoiksi kirjaimiksi (ylös)
\end{verbatim}

Formaatin lopussa voi olla jompikumpi kirjaimista \verb=NL=.

\begin{verbatim}
N perusmuodoksi hyväksytään vain nimisana tai nimilaatusana
L perusmuodoksi hyväksytään vain laatusana tai nimilaatusana
\end{verbatim}

Tämä on tarpeen, koska eräät taivutusmuodot voivat olla sekä nimisanan
että inen-loppuisen laatusanan taivutusmuotoja, esimerkiksi

\verb|talvista => talvi, talvinen|


\verb=\VXL= on sama asia kuin \verb|\VXF{<lista>}{p=,s}[<välimerkki>]|

mutta <lista>ssa voi olla vain kaksi sanaa. Esimerkiksi

\begin{verbatim}
\VXL{Helsingin nuorisoseuralle}[.] =>
Helsingin\vxp{Helsinki} nuorisoseuralle.\vxs{Helsingin nuorisoseura}
\end{verbatim}


\subsection*{Yksikkö ja monikko}

Voikko muuttaa kaikki nominit yksikön nominatiiviin. Esimerkiksi
Yhdysvalloista tulee Yhdysvalta. Tätä varten voikkoindexissä on kolme
komentoa. Kaikilla on kaksi parametria, joista ensimmäinen on Voikon
palauttama arvo tai voikkoindexiin lisätty sana (katso seuraava
kohta), ja toinen on sana (jonka ei tarvitse olla monikko), joka
halutaan laittaa hakemistoon. Esimerkiksi

\begin{verbatim}
\SetSurnameIndex{Wirtanen}{Virtanen}
\SetPlaceNameIndex{Yhdysvalta}{Yhdysvallat}
\SetWordIndex{nuorisoseura}{nuorisoseurat}
\end{verbatim}


\subsection*{Voikosta puuttuvat sanat}

Voikosta puuttuvat sanat voidaan lisätä voikkoindexiin tällä tavalla

\begin{verbatim}
\SetExtraSurname{Hyvämä|ki}
\SetExtraPlaceName{Takahikiä}
\SetExtraWord{teikar|i}
\end{verbatim}

Koska sanat etsitään luettelosta siinä järjestyksessä, kun ne on
siihen laitettu, jos lisättävä sana on sama kuin pitemmän sanan
alkuosa, pitempi sana pitää olla ensin. Esimerkiksi

\begin{verbatim}
\SetExtraWord{qwertyuiop}
\SetExtraWord{qwerty}
\end{verbatim}

Muuten pitemmän sanan asemesta hakemistoon lisätään aina lyhyempi
sana.

Koska nämä sanat eivät siis ole Voikossa, niitä ei voi muuttaa
perusmuotoon, mutta jos hakemistoon lisättävä merkkijono alkaa samalla
tavalla kuin jokin tällä tavalla lisätyistä sanoista, sana lisätään
hakemistoon. Jos lisätyssä sanassa on pystyviiva, sanaa tutkitaan vain
siihen asti. Esimeriksi kaikki \verb=teikar=-alkuiset merkkijonot
lisätään asiasanahakemistoon muodossa \verb=teikari= ja kaikki
\verb=Hyvämä=-alkuiset sukunimet nimihakemistoon muodossa
\verb=Hyvämäki=. (Mutta katso myös edellinen kohta.)

Tämäkin toimii

\begin{verbatim}
\SetExtraSurname{Wirta|nen}
\SetSurnameIndex{Wirtanen}{Virtanen}
\end{verbatim}


\subsection*{Mitä tarvitaan ja mistä ne saa?}

\begin{itemize}
\item voikkoindex: \\
      \verb=https://github.com/ahomansikka/voikkoindex=
      Koska luet tätä tekstiä, olet jo imuroinut tämän. (-:

\item corevoikko: \\
      \verb=https://github.com/voikko/corevoikko=
      Uusin versio. Sen pitää olla käännetty niin, että perusmuodot
      tulevat mukaan.

\item lualibvoikko: \\
      \verb=https://github.com/ahomansikka/lualibvoikko=

\item lualatex: \\
      Tulee ainakin TeXLiven mukana.

\item splitidx: \\
      Tulee ainakin TeXLiven mukana.

\item luarocks: \\
      Se voidaan asentaa komennolla
      \verb=sudo apt istall luarocks=

\item luautf8: \\
      \url{https://github.com/starwing/luautf8}
      Se voidaan asentaa komennolla
      \verb=sudo luarocks install luautf8=
\end{itemize}


Jos lualatex ei löydä luautf8:aa, pitää määritellä ympäristömuuttuja
(esim. tiedostossa .bashrc), joka kertoo missä tiedosto
\verb=lua-utf8.so= on:

\verb|export CLUAINPUTS=/usr/local/lib/lua/5.3//|

Sen jälkeen voikkoindex'iä voi käyttää komennolla lualatex, esimerkiksi

\begin{verbatim}
lualatex tiedosto.tex
splitindex <parametrit> tiedosto.idx
lualatex tiedosto.tex
\end{verbatim}


\newpage \printindex[nimet]
\newpage \printindex[paikat]
\newpage \printindex[sanat]
\end{document}
